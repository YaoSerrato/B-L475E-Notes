%%% DOCUMENTCLASS 
%%%-------------------------------------------------------------------------------

\documentclass[
a4paper, % Stock and paper size.
11pt, % Type size.
% article,
% oneside, 
onecolumn, % Only one column of text on a page.
% openright, % Each chapter will start on a recto page.
% openleft, % Each chapter will start on a verso page.
openany, % A chapter may start on either a recto or verso page.
]{memoir}

%%% PACKAGES 
%%%------------------------------------------------------------------------------

\usepackage[utf8]{inputenc} % If utf8 encoding
% \usepackage[lantin1]{inputenc} % If not utf8 encoding, then this is probably the way to go
\usepackage[T1]{fontenc}    %
\usepackage{lmodern}
\usepackage[english]{babel} % English please
\usepackage[final]{microtype} % Less badboxes

% \usepackage{kpfonts} %Font

\usepackage{amsmath,amssymb,mathtools} % Math

% \usepackage{tikz} % Figures
\usepackage{graphicx} % Include figures

%%% PAGE LAYOUT 
%%%------------------------------------------------------------------------------

\setlrmarginsandblock{0.15\paperwidth}{*}{1} % Left and right margin
\setulmarginsandblock{0.2\paperwidth}{*}{1}  % Upper and lower margin
\checkandfixthelayout

%%% SECTIONAL DIVISIONS
%%%------------------------------------------------------------------------------

\maxsecnumdepth{subsection} % Subsections (and higher) are numbered
\setsecnumdepth{subsection}

\makeatletter %
\makechapterstyle{standard}{
  \setlength{\beforechapskip}{0\baselineskip}
  \setlength{\midchapskip}{1\baselineskip}
  \setlength{\afterchapskip}{8\baselineskip}
  \renewcommand{\chapterheadstart}{\vspace*{\beforechapskip}}
  \renewcommand{\chapnamefont}{\centering\normalfont\Large}
  \renewcommand{\printchaptername}{\chapnamefont \@chapapp}
  \renewcommand{\chapternamenum}{\space}
  \renewcommand{\chapnumfont}{\normalfont\Large}
  \renewcommand{\printchapternum}{\chapnumfont \thechapter}
  \renewcommand{\afterchapternum}{\par\nobreak\vskip \midchapskip}
  \renewcommand{\printchapternonum}{\vspace*{\midchapskip}\vspace*{5mm}}
  \renewcommand{\chaptitlefont}{\centering\bfseries\LARGE}
  \renewcommand{\printchaptertitle}[1]{\chaptitlefont ##1}
  \renewcommand{\afterchaptertitle}{\par\nobreak\vskip \afterchapskip}
}
\makeatother

\chapterstyle{standard}

\setsecheadstyle{\normalfont\large\bfseries}
\setsubsecheadstyle{\normalfont\normalsize\bfseries}
\setparaheadstyle{\normalfont\normalsize\bfseries}
\setparaindent{0pt}\setafterparaskip{0pt}

%%% FLOATS AND CAPTIONS
%%%------------------------------------------------------------------------------

\makeatletter                  % You do not need to write [htpb] all the time
\renewcommand\fps@figure{htbp} %
\renewcommand\fps@table{htbp}  %
\makeatother                   %

\captiondelim{\space } % A space between caption name and text
\captionnamefont{\small\bfseries} % Font of the caption name
\captiontitlefont{\small\normalfont} % Font of the caption text

\changecaptionwidth          % Change the width of the caption
\captionwidth{1\textwidth} %

%%% ABSTRACT
%%%------------------------------------------------------------------------------

\renewcommand{\abstractnamefont}{\normalfont\small\bfseries} % Font of abstract title
\setlength{\absleftindent}{0.1\textwidth} % Width of abstract
\setlength{\absrightindent}{\absleftindent}

%%% HEADER AND FOOTER 
%%%------------------------------------------------------------------------------

\makepagestyle{standard} % Make standard pagestyle

\makeatletter                 % Define standard pagestyle
\makeevenfoot{standard}{}{}{} %
\makeoddfoot{standard}{}{}{}  %
\makeevenhead{standard}{\bfseries\thepage\normalfont\qquad\small\leftmark}{}{}
\makeoddhead{standard}{}{}{\small\rightmark\qquad\bfseries\thepage}
% \makeheadrule{standard}{\textwidth}{\normalrulethickness}
\makeatother                  %

\makeatletter
\makepsmarks{standard}{
\createmark{chapter}{both}{shownumber}{\@chapapp\ }{ \quad }
\createmark{section}{right}{shownumber}{}{ \quad }
\createplainmark{toc}{both}{\contentsname}
\createplainmark{lof}{both}{\listfigurename}
\createplainmark{lot}{both}{\listtablename}
\createplainmark{bib}{both}{\bibname}
\createplainmark{index}{both}{\indexname}
\createplainmark{glossary}{both}{\glossaryname}
}
\makeatother                               %

\makepagestyle{chap} % Make new chapter pagestyle

\makeatletter
\makeevenfoot{chap}{}{\small\bfseries\thepage}{} % Define new chapter pagestyle
\makeoddfoot{chap}{}{\small\bfseries\thepage}{}  %
\makeevenhead{chap}{}{}{}   %
\makeoddhead{chap}{}{}{}    %
% \makeheadrule{chap}{\textwidth}{\normalrulethickness}
\makeatother

\nouppercaseheads
\pagestyle{standard}               % Choosing pagestyle and chapter pagestyle
\aliaspagestyle{chapter}{chap} %

%%% NEW COMMANDS
%%%------------------------------------------------------------------------------

\newcommand{\p}{\partial} %Partial
% Or what ever you want

%%% TABLE OF CONTENTS
%%%------------------------------------------------------------------------------

\maxtocdepth{subsection} % Only parts, chapters and sections in the table of contents
\settocdepth{subsection}

\AtEndDocument{\addtocontents{toc}{\par}} % Add a \par to the end of the TOC

%%% INTERNAL HYPERLINKS
%%%-------------------------------------------------------------------------------

\usepackage{hyperref}   % Internal hyperlinks
\hypersetup{
pdfborder={0 0 0},      % No borders around internal hyperlinks
pdfauthor={I am the Author} % author
}
\usepackage{memhfixc}   %

%%% THE DOCUMENT
%%% Where all the important stuff is included!
%%%-------------------------------------------------------------------------------

\author{Yaoctzin Serrato}
\title{STM32L475 Embedded Driver Development}

\usepackage{lipsum} % Just to put in some text

\begin{document}

\frontmatter

\maketitle

\begin{abstract}
\lipsum[1-2]
\end{abstract}
\clearpage

\tableofcontents*
\clearpage

\chapter{USART driver}

UART stands for Universal Asynchronous Receiver Transmitter, whereas USART stands for Universal Synchronous Asynchronous Receiver Transmitter. They are basically a piece of hardware that converts parallel data to serial data. The only difference is that UART supports only Asynchronous mode, whereas USART supports both synchronous and asynchronous modes.

\noindent Unlike Ethernet or USB, there is no specific port for UART/USART communication. They are commonly used in conjunction with protocols such as RS232, RS434, etc.

\noindent In synchronous transmission the clock signal is sent separately from the data stream and no start/stop bits are used. If it is asynchronous mode, then the clock signal is not sent but instead, synchronization bits will be used: start and stop bits, besides the data stream.

\subsubsection{USART hardware components}
Typically, USART hardware will have the following components:

\begin{itemize}
	\item	Baudrate generator
	\item	TX/RX shift registers
	\item	Transmit/Receive control blocks
	\item	Transmit/Receive buffers
	\item	First-in First-out (FIFO) buffer memory
\end{itemize}

\subsubsection{USART pins}
USART bidirectional communication requires the following pins:

\begin{itemize}
	\item	\textbf{TX - Transmit (required)}: USART module transmits data over TX pin. If nothing is beaing transmitted, then the TX line will remain HIGH, which is the idle state of TX line.
	\item	\textbf{RX - Receive (required)}: USART module receives data over RX pin. The module continuously samples RX line to detect the start bit of an incoming frame.
	\item	\textbf{RTS - Request to Send (optional)}: RTS pin is optional for basic communication set up, but required if hardware flow control is used. RTS is an active-LOW pin. This pin is used by the USART module to inform an external device that new data is needed (RTS pin is then pulled to LOW).
	\item	\textbf{CTS - Clear to Send (optional)}: CTS pin is optional for basic communication set up, but required if hardware flow control is used. CTS is an active-LOW pin. When hardware flow control is used, the data transmision on the TX line happens only if the CTS pin is pulled LOW. Otherwise, the data transmision will remain inactive. CTS pin has to be pulled LOW by another device in order to enable the data transmision over TX pin.
\end{itemize}

\noindent RX pin of one device is connected to TX pin of another device. The same happens with CTS and RTS pins: RTS pin of one device is connected to CTS pin of another device. If device A wants data from device B, and hardware flow control is being used, device A will pull LOW its RTS pin which in turn will pull LOW the CTS pin of device B and the data frame will be transmitted over TX line of device B and received over RX line of device A.

\begin{center}
\textit{Image of USART pins (Microsoft VISIO)}
\end{center}

\subsubsection{USART frame format}
A frame refers to the entire data packet which is being sent or received during the communication. The formats of the data packet vary from protocol to protocol. The following image shows the frame format of USART packets.

\begin{center}
\textit{Image of USART frame formats for 9-bit and 8-bit word lengths (Microsoft VISIO)}
\end{center}

USART frame commences with a LOW start bit of 1-bit duration. Then follows data bits from LSB to MSB (although this can be configured to start with MSB and finish with LSB) and this bit stream has a length of 5 to 9 bits. The data bits are followed by the parity bit, which is an optional bit. If parity bit is used, it will occupy one bit from the data stream, which means, for a 9-bit packet 8 bits will be of data plus the parity bit; for an 8-bit packet 7 bits will be of data plus the parity bit. Parity can be either odd or even. Finally, a frame ends with an stop bit. Stop bit is always HIGH and can be configured for 1, 1.5 or 2 bit duration.

\subsubsection{USART baudrate}
The significance of baudrate is how fast the data is sent over a serial line. It is usually expressed in units of bits-per-second $[bps]$. If the baudrate value is inverted, the result is the time a single bit takes to be transmitted. For example, a bit takes $104 [us]$ to be transmitted with a baudrate of $9600 [bps]$. Both transmitting and receiving devices should operate at the same rate to have a proper communication. The higher the baudrate goes, the faster the data is sent or received. The baudrates are usually depending on the peripheral clock frequency of the USART peripheral.

\subsubsection{USART synchronization bits}
The synchronization bits are 2 to 3 bits that are transfered in each frame of data. These bits are the start bits and stop bits and they mark the begining and end of a packet, respectively. There is always 1 start bit but stop bits are configurable to 1, 1.5 or 2 bits. Typically, 1 stop bit is used, however, if the baud rate is high (in $MHz$), then it is recommended to use 2 stop bits.

\noindent The start bit is always indicated by an idle data line going from HIGH to LOW, while the stop bits will transition back to the idle state (HIGH value). The receive engine of the USART module is capable enough for catching these transitions using over sampling techniques.

\begin{center}
\textit{Image of USART frame formats showing start and stop bits (Microsoft VISIO)}
\end{center}

\subsubsection{USART parity bit}
Adding a parity bit is the simplest method of error detection. Parity is simply the number of ones (1) appearing in the binary form of a number. For example, the binary representation of the decimal value 55 is 0b00110111, which has 5 ones.

\mainmatter

% \chapter{How to Build a Timemachine}

% \lipsum[1-13] 

% \chapter{How to Destroy a Timemachine}

% \lipsum[1-14]

% \appendix

% \chapter{Causality}

% \lipsum[1-15]

% \backmatter

%%% BIBLIOGRAPHY
%%% -------------------------------------------------------------

% \bibliographystyle{utphysics}
% \bibliography{ref}

\end{document}